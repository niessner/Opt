\begin{abstract}
Many graphics and vision problems are naturally phrased as optimizations with (either linear or) nonlinear least squares objective functions over visual data, like images.
The mathematical descriptions of these functions is extremely concise, but their implementation in real code is anything but, especially when optimized for real-time performance, either for interactive applications or streaming live capture.
We propose a new language in which a user simply writes energy functions over pixel-structured unknowns, and a compiler automatically generates state-of-the-art GPU optimization kernels.
The end result is a system in which real-world energy functions in graphics and vision applications are expressible in tens of lines of code, but compile directly into highly optimized GPU solver implementations with performance competitive with the best published hand-tuned, application-specific GPU solvers.
\end{abstract}

% the time from energy function concept to optimized GPU implementation, with the potential for real-time performance, is shortened from hours or days to minutes
% [It would be nice if libraries could do this, but there are two problems…]
  % solvers don't need the energy function, but some derivatives thereof
    % which derivatives varies by solver method
    % derivatives are much messier than the energy function, itself
      % error prone manual derivation
      % much more complex code
      % re-derivation to work with different solver methods
  % traditional subroutine (library) composition is high overhead / prohibitively inefficient to match the performance of hand-tuned application-specific solvers in these cases
    % hand-tuned solvers dominate in performance-critical apps [cite], but in many cases this is not because the solver methods must be customized to the app, but rather just because the solver method and application energy function/data structures must be tightly intertwined to remove overhead
